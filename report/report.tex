\documentclass[14pt, openany, titlepage]{report} % Explicitly use openany and titlepage
\usepackage[utf8]{inputenc}
\usepackage[italian,english]{babel}
\usepackage{graphicx}
\usepackage{cite}
\usepackage{amsmath}
\usepackage[table,xcdraw]{xcolor}
\usepackage[italian]{minitoc}
\usepackage{fancybox}
\usepackage{fancyhdr}
\usepackage{verbatim}
\usepackage{url}
\usepackage{color}
\usepackage{listings}
\usepackage{makeidx}
\usepackage{comment}
\usepackage{hyperref}
\usepackage{bm}
\usepackage{longtable}
\usepackage{tabularx}
\usepackage{placeins}
\usepackage{caption}
\usepackage{float}
\usepackage{xcolor}
\usepackage{soul}
\usepackage{array}
\usepackage{multirow}
\usepackage{adjustbox} % Pacchetto per ridimensionare automaticamente
\usepackage{graphicx} % Per il ridimensionamento
\usepackage{lscape}
\usepackage{mathtools}
\usepackage{amsmath}
\usepackage{amssymb}
\usepackage{listings}

% Set link colors without borders
\hypersetup{
    colorlinks=true,
    linkcolor=black,
    filecolor=magenta,      
    urlcolor=cyan,
}

\lstset{frame=tb,
    language=Java,
    numbers=left,
    keywordstyle=\color{blue},
    alsoletter={.}
}
\graphicspath{ {./Figure/} }


% Minimize additional blank pages
\let\cleardoublepage\clearpage % Suppresses new page on odd side
\flushbottom % Avoid vertical blank space between paragraphs

% Title Page
\begin{document}
\selectlanguage{italian}

\begin{titlepage}
\begin{center}
    \begin{figure}
        \includegraphics[width=3.5cm, height=3.5cm]{unisa.png}
        \centering
    \end{figure}
    {\Large Università degli Studi di Salerno}\\[0.2truecm]
    {\large Dipartimento di Informatica\\Corso di Laurea Triennale in Informatica}\\
    \hrulefill
    \vfill
    {\large Progetto Calcolo Probabilità Statistica Matematica\\(CPSM)}\\[0.1truecm]
    \vfill\vfill
    {\LARGE {\bf Indagine Statistica sulla\\[0.1truecm]Ricerca di Lavoro tra Laureati Italiani}}
    \vfill\vfill
    
    \hfill  \textbf{Tozza Gennaro Carmine}
    \centerline{\hfill Matricola: 0512120382}
    
    \vfill
    \hrulefill 
    \begin{center} Anno Accademico 2024-2025 \end{center}
\end{center}
\end{titlepage}

% Suppress page break after table of contents
\tableofcontents

\chapter{Introduzione}
Il progetto consiste nello sviluppo di un software che analizzi i dati statistici relativi alla ricerca di lavoro tra i laureati italiani che hanno conseguito il titolo nel 2011, con dati raccolti nel 2015. \\\\
\footnote{\url{https://siqual.istat.it/SIQual/visualizza.do?id=0032700&refresh=true&language=IT}}L’indagine è rivolta a un campione di laureati e approfondisce la loro condizione e il loro percorso occupazionale a distanza di alcuni anni dal conseguimento del titolo. \\\\
Rileva informazioni sulla tipologia di attività lavorativa svolta dai laureati, sulla professione, sulla retribuzione e sulla loro soddisfazione per il lavoro svolto, sul settore in cui lavorano e sull'utilizzo nel lavoro delle competenze acquisite all'università.\\\\
L’indagine fa parte del sistema di rilevazioni sulla transizione istruzione-lavoro, che comprende anche le indagini sull'inserimento professionale dei dottori di ricerca e sui percorsi di studio e di lavoro dei diplomati di scuola secondaria di II grado.\\\\







\end{document}